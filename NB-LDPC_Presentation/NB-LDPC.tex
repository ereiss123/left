\PassOptionsToPackage{unicode=true}{hyperref} % options for packages loaded elsewhere
\PassOptionsToPackage{hyphens}{url}
%
\documentclass[10pt,ignorenonframetext,]{beamer}
\setbeamertemplate{caption}[numbered]
\setbeamertemplate{caption label separator}{: }
\setbeamercolor{caption name}{fg=normal text.fg}
\beamertemplatenavigationsymbolsempty
\usepackage{lmodern}
\usepackage{amssymb,amsmath}
\usepackage{ifxetex,ifluatex}
\usepackage{fixltx2e} % provides \textsubscript
\ifnum 0\ifxetex 1\fi\ifluatex 1\fi=0 % if pdftex
  \usepackage[T1]{fontenc}
  \usepackage[utf8]{inputenc}
  \usepackage{textcomp} % provides euro and other symbols
\else % if luatex or xelatex
  \usepackage{unicode-math}
  \defaultfontfeatures{Ligatures=TeX,Scale=MatchLowercase}
\fi
% use upquote if available, for straight quotes in verbatim environments
\IfFileExists{upquote.sty}{\usepackage{upquote}}{}
% use microtype if available
\IfFileExists{microtype.sty}{%
\usepackage[]{microtype}
\UseMicrotypeSet[protrusion]{basicmath} % disable protrusion for tt fonts
}{}
\IfFileExists{parskip.sty}{%
\usepackage{parskip}
}{% else
\setlength{\parindent}{0pt}
\setlength{\parskip}{6pt plus 2pt minus 1pt}
}
\usepackage{hyperref}
\hypersetup{
            pdftitle={Non-Binary LDPC Code Construction and Decoding},
            pdfauthor={Eric Reiss},
            pdfborder={0 0 0},
            breaklinks=true}
\urlstyle{same}  % don't use monospace font for urls
\newif\ifbibliography
\usepackage{longtable,booktabs}
\usepackage{caption}
% These lines are needed to make table captions work with longtable:
\makeatletter
\def\fnum@table{\tablename~\thetable}
\makeatother
% Prevent slide breaks in the middle of a paragraph:
\widowpenalties 1 10000
\raggedbottom
\setbeamertemplate{part page}{
\centering
\begin{beamercolorbox}[sep=16pt,center]{part title}
  \usebeamerfont{part title}\insertpart\par
\end{beamercolorbox}
}
\setbeamertemplate{section page}{
\centering
\begin{beamercolorbox}[sep=12pt,center]{part title}
  \usebeamerfont{section title}\insertsection\par
\end{beamercolorbox}
}
\setbeamertemplate{subsection page}{
\centering
\begin{beamercolorbox}[sep=8pt,center]{part title}
  \usebeamerfont{subsection title}\insertsubsection\par
\end{beamercolorbox}
}
\AtBeginPart{
  \frame{\partpage}
}
\AtBeginSection{
  \ifbibliography
  \else
    \frame{\sectionpage}
  \fi
}
\AtBeginSubsection{
  \frame{\subsectionpage}
}
\setlength{\emergencystretch}{3em}  % prevent overfull lines
\providecommand{\tightlist}{%
  \setlength{\itemsep}{0pt}\setlength{\parskip}{0pt}}
\setcounter{secnumdepth}{0}

% set default figure placement to htbp
\makeatletter
\def\fps@figure{htbp}
\makeatother


\title{Non-Binary LDPC Code Construction and Decoding}
\providecommand{\subtitle}[1]{}
\subtitle{Based on ``Low-Density Parity Checks over GF(q)'' by M. Davey
and D. Mackay}
\author{Eric Reiss}
\date{}


\usepackage[most]{tcolorbox}

\tcbset{
  frame code={},
  center title,
  left=0pt,
  right=0pt,
  top=0pt,
  bottom=0pt,
  colback=blue!20,
  colframe=white,
  width=\dimexpr\textwidth\relax,
  enlarge left by=0mm,
  boxsep=5pt,
  arc=0pt,outer arc=0pt,
}


\pgfdeclareimage[]{mybackground}{figures/OldMainTower.png}

\setbeamertemplate{title page}{

  \begin{picture}(0,0)
    
     \put(50,-100){%
      \pgfuseimage{mybackground}
    }
     
    \put(0,-40.7){%
      \begin{minipage}[b][45mm][t]{226mm}
        \usebeamerfont{title}{\inserttitle\par}
        \usebeamerfont{subtitle}{\insertsubtitle\par}
        \vspace{0.5cm}
        \usebeamerfont{author}{\insertauthor\par}
        \usebeamerfont{author}{Utah State University\par}
      \end{minipage}
    }
    
  \end{picture}
}


\setbeamertemplate{itemize/enumerate subbody begin}{\vspace{0.125cm}\begin{tcolorbox}[colback=red!20]}
\setbeamertemplate{itemize/enumerate subbody end}{\end{tcolorbox}\vspace{0.125cm}}
\setbeamertemplate{itemize/enumerate body begin}{\vspace{0.125cm}\begin{tcolorbox}}
\setbeamertemplate{itemize/enumerate body end}{\end{tcolorbox}\vspace{0.125cm}}
\setbeamertemplate{itemize/enumerate item end}{\vspace{0.125cm}}

\setlength{\itemsep}{0.5cm}

\makeatletter
\addtobeamertemplate{itemize begin}{
\def\@listi{\leftmargin\leftmargini
              \topsep    0pt
              \parsep    0pt
              \itemsep   3pt plus 2pt minus 3pt}
\partopsep 0pt
}
\makeatother



\begin{document}
\frame{\titlepage}

\begin{frame}{Overview}
\protect\hypertarget{overview}{}
\end{frame}

\begin{frame}{Term Definition}
\protect\hypertarget{term-definition}{}
\begin{itemize}[<+->]
\tightlist
\item
  weight - number of non-zero elements in a vector or matrix
\item
  density - expected fraction of non-zero symbols in a source of random
  symbols
\item
  overlap - number of cooridinates in which two vectors have non-zero
  entries
\item
  \(H\) - sparse random parity check matrix
\item
  \(N\) - transmitted block length
\item
  \(K\) - source block length
\item
  \(M\) - number of parity checks, \(M=N-K\)
\item
  \(t\) - mean column weight
\item
  symbol - element of \(GF(2^b)\)
\item
  bit - binary representation of a symbol
\end{itemize}
\end{frame}

\begin{frame}{Code Construction}
\protect\hypertarget{code-construction}{}
\begin{itemize}[<+->]
\tightlist
\item
  Let \(H\) be an \(M\times N\) random parity check matrix

  \begin{itemize}[<+->]
  \tightlist
  \item
    The weight per column will be greater than 2 with a mean of \(t\)
  \item
    The weight per row will be uniform as possible
  \end{itemize}
\item
  Non-zero elements are selected from a \textbf{special distribution} to
  maximize entropy of syndrome

  \begin{itemize}[<+->]
  \tightlist
  \item
    Each codeword should have roughly the same likelihood
  \item
    Citation by \textbf{special distribution} is for a work in progress
    paper that I don't believe was published, at least not under that
    name
  \end{itemize}
\item
  Generator matrix is obtained through gaussian elimination on \(H\)
\item
  If rows of \(H\) are not independent, then \(H\) defines a parity
  check for the same \(N\) but a smaller \(M\)

  \begin{itemize}[<+->]
  \tightlist
  \item
    \(H\) defines a code of at least \(K/N\)
  \end{itemize}
\end{itemize}
\end{frame}

\begin{frame}{Channel Model}
\protect\hypertarget{channel-model}{}
\begin{itemize}[<+->]
\tightlist
\item
  Used a memoryless, binary symmetric and binary gaussian channel with
  inputs \(\pm s\) and variance \(\sigma^2=1\)
\item
  For a code-rate \(R\), \(SNR=\frac{s^2}{2R\sigma^2}\) and if
  \(\sigma^2=1\) then \(SNR=\frac{s^2}{2R}\)
\item
  For \(GF(2^b)\), \(\textbf{x}\) is a sample from the assumed noise
  model consisting of noise symbols \(x_n\), which in turn consist of
  \(b\) bits
\item
  The received bit is assigned to be the sign of the output
\item
  The likelihood that the nth noise bit is 1 is given by
  \(g_n^1=\frac{1}{1+e^{2s|y_n|/\sigma^2}}\) where \(y_n\) is the
  channel output

  \begin{itemize}[<+->]
  \tightlist
  \item
    The likelihood the bit is 0 is \(1-g_n^1\)
  \end{itemize}
\item
  The likelihood that \(x_n=a\), \(a\in GF(2^b)\) is defined
  \(f^a_n:=\prod_{i=1}^b g^{a_i}_{n_i}\) where \(a_i\) is the ith bit of
  the binary representation of \(a\)

  \begin{itemize}[<+->]
  \tightlist
  \item
    Ex. Likelihood \(x_1=1\) in \(GF(4)\):
    \(f^1_1 = \prod_{i=1}^2 g^{a_i}_{1_i} = g^{0}_{1_1} * g^1_{1_2} = (1-\frac{1}{e^{2s|y_{1_1}|/\sigma^2}}) * \frac{1}{e^{2s|y_{1_2}|/\sigma^2}}\)
  \end{itemize}
\end{itemize}
\end{frame}

\begin{frame}{Decoding Algorithm}
\protect\hypertarget{decoding-algorithm}{}
\begin{itemize}[<+->]
\tightlist
\item
  Objective: find most probable vector \(\textbf{x}\) s.t.
  \(\textbf{Hx}=z\)
\item
  Elements of \(\textbf{x}\) are referred to as noise symbols
\item
  Elements of \(\textbf{z}\) are referred to as checks
\item
  Define \(\mathcal{N}(m) = \{n:H_{mn}\neq 0\}\) as the set of symbols
  nodes, \(n\), adjacent to check node \(m\)
\item
  Define \(\mathcal{M}(n) = \{m:H_{mn}\neq 0\}\) as the set of check
  nodes, \(m\), adjacent to symbol node \(n\)
\item
  For each non-zero entry in \(H\), define \(q_{mn}^a\) and \(r_{mn}^a\)
  for \(a\in GF(2^b)\)

  \begin{itemize}[<+->]
  \tightlist
  \item
    \(q_{mn}^a\) is the probability that symbol \(n\) of \(\textbf{x}\)
    is \(a\)
  \item
    \(r_{mn}^a\) is the probability that check \(m\) is satisfied if
    symbol \(n\) of \(\textbf{x}\) is fixed at \(a\)
  \end{itemize}
\end{itemize}
\end{frame}

\begin{frame}{Algorithm}
\protect\hypertarget{algorithm}{}
\begin{itemize}[<+->]
\tightlist
\item
  Initialize \(q_{mn}^a\) to \(f^a_n\)
\item
  Update \(r_{mn}^a\) as
  \(r_{mn}^a = \sum\limits_{\textbf{x'}:x'_n=a}\text{Prob}[z_m|\textbf{x'}]\prod\limits_{j\in\mathcal{N}(m)/n}q^{x'_j}_{mj}\)

  \begin{itemize}[<+->]
  \tightlist
  \item
    \(\text{Prob}[z_m|\textbf{x'}]\in [0,1]\) depending on if
    \(\textbf{x'}\) satisfies check \(m\)
  \item
    Davey and Mackay introduce some simplifications
  \item
    Define \(\sigma_{mk} := \sum_{j:j\le k} H_{mj}x'_j\)
  \item
    Define \(\rho_{mk} := \sum_{j:j\ge k} H_{mj}x'_j\)
  \item
    Calculate Prob\([\sigma_{mk}=a]\) for each \(a\in GF(2^b)\) and each
    \(k\in \mathcal{N}(m)\)
  \item
    Prob\([\sigma_{mk}=a]\) =
    \(\sum\limits_{s,t:H_{mj}t+s=a}\text{ Prob}[\sigma_{mi}=s]q^t_{mj}\)
    if \(i,j\) are successive and \(j>i\)
  \item
    \(\rho_{mk}\) is calculated in a similiar way
  \item
    Then
    \(r^a_{mn} = \text{ Prob}[(\sigma_{m(n-1)}+\rho_{m(n-1)})=z_m-H_{mn}a]\)
  \item
    Expanded as
    \(r^a_{mn} = \sum\limits_{s,t:s+t=z_m-H_{mn}a}\text{ Prob}[\sigma_{m(n-1)=s}] * \text{Prob}[\rho_{m(n+1)}=t]\)
  \end{itemize}
\end{itemize}
\end{frame}

\begin{frame}{Algorithm Cont.}
\protect\hypertarget{algorithm-cont.}{}
\begin{itemize}[<+->]
\tightlist
\item
  Update \(q^a_{mn}\)

  \begin{itemize}[<+->]
  \tightlist
  \item
    Define \(\alpha_{mn}\) as a weight
  \item
    \(q^a_{mn} = \alpha_{mn} f^a_n\prod\limits_{j\in\mathcal{M}(n)\\m}r^a_{jn}\)
  \item
    Select \(\alpha_{mn}\) s.t. \(\sum_{a=1}^q q^a_{mn} = 1\)
  \end{itemize}
\item
  Make tentative decoding
  \(\hat{x_n} = \text{argmax}(a)f^a_n\prod\limits_{j\in\mathcal{M}(n)}r^a_{jn}\)

  \begin{itemize}[<+->]
  \tightlist
  \item
    If \(H\hat{x} = z\) then the algorithm is complete
  \item
    Else it repeats until a valid decoding is obtained or maximum number
    of iterations is met
  \end{itemize}
\end{itemize}
\end{frame}

\begin{frame}{GF(4)}
\protect\hypertarget{gf4}{}
\begin{itemize}[<+->]
\tightlist
\item
  Explicitly defined below

  \begin{itemize}[<+->]
  \item
    \begin{longtable}[]{@{}lllll@{}}
    \toprule
    \textbf{\(\oplus\)} & \textbf{a} & \textbf{b} & \textbf{c} &
    \textbf{d} \\
    \midrule
    \endhead
    \textbf{a} & a & b & c & d \\
    \textbf{b} & b & c & d & a \\
    \textbf{c} & c & d & a & b \\
    \textbf{d} & d & a & b & c \\
    \bottomrule
    \end{longtable}
  \item
    \begin{longtable}[]{@{}lllll@{}}
    \toprule
    \textbf{*} & \textbf{a} & \textbf{b} & \textbf{c} & \textbf{d} \\
    \midrule
    \endhead
    \textbf{a} & a & a & a & a \\
    \textbf{b} & a & b & c & d \\
    \textbf{c} & a & c & d & c \\
    \textbf{d} & a & d & c & d \\
    \bottomrule
    \end{longtable}
  \end{itemize}
\end{itemize}
\end{frame}

\begin{frame}{Sources}
\protect\hypertarget{sources}{}
\begin{itemize}[<+->]
\tightlist
\item
  {[}1{]} M. C. Davey and D. MacKay, ``Low-density parity check codes
  over GF(q),'' in IEEE Communications Letters, vol.~2, no. 6,
  pp.~165-167, June 1998, doi: 10.1109/4234.681360.
\end{itemize}
\end{frame}

\end{document}
